\documentclass[14pt, a4paper, titlepage, fleqn]{extarticle}
\usepackage[english,russian]{babel}
\usepackage[utf8x]{inputenc}
\usepackage{amsmath}
\usepackage{cmap}
\usepackage{listings}


\author{Никифоров Данил}
\title{Лабораторная работа №1}
\date{\today}



\begin{document}
	


    \maketitle
    	
    \tableofcontents
    
    \pagebreak
    	
    \section*{Введение}
    \addcontentsline{toc}{section}{Введение}
    \textit{Цель:} научиться верстать в LaTeX
    	
    
        
    \section*{Задания}
    \addcontentsline{toc}{section}{Задания}
    
    \subsection*{1 задание}
    \addcontentsline{toc}{subsection}{1 задание}
    Вычислить следующий интеграл с подробным описанием всех действий:
    \[\int x \cdot{\ln{4x}} ~ dx \]
    
    \begin{itemize}
    
        \item Воспользуемся формулой интегрирования по частям: \[\int u ~ dv = uv - \int v ~ du\]
        \[\begin{aligned} 
        	& \int x \cdot{\ln{4x}} ~ dx = (*) \left\vert{
        	\begin{matrix}
        	
        		u = \ln{4x}, & du = \frac{~dx}{x} \\ 
        		\\
        		dv = x ~dx, & v = \frac{x^2}{2}
        	
        	\end{matrix}} \right\vert (*) = \ln{4x} \cdot{\frac{x^2}{2}} - \int{\frac{x^2}{2} \cdot \frac{~dx}{x}} = \\\\
        &= \ln{4x} \cdot{\frac{x^2}{2}} - \frac{1}{2} \cdot \int{x ~dx} = \ln{4x} \cdot \frac{x^2}{2} - \frac{1}{4} \cdot{x^2} + C.
        \end{aligned}\]
	   
    \end{itemize}
    		
    \subsection*{2 задание}
    \addcontentsline{toc}{subsection}{2 задание}
    Численно вычислить следующий интеграл с точностью $\varepsilon  = 10^{-5}$ :
    \[\int_{0}^{\infty} {\sin{x^2}} ~ dx = 0.626657\]
    
    \pagebreak
    Код для вычисления данного интеграла.
    \begin{lstlisting}[language=C++, basicstyle=\footnotesize\ttfamily, frame=single]
#include <iostream>
#include <math.h>
#include <vector>
using namespace std;

double f(double x) { return sin(x*x); }

int main() {
   vector <double> Integral(4);
   double a = 0.0, b = 2000;
   double h = 0.00001;
   double n = (b - a) / h;

   for(int i = 0; i < n; i++) {
      Integral[0] += f(a + h * i) * h;//left
      Integral[1] += f((a + h/2) + h * i) * h;
      if (i > 0) {
         Integral[2] += f(a + h * i) * h;
      }
      Integral[3] += (f(a + h * i) + f(a + h * (i + 1))) / 2 * h;
   }
   Integral[2] += f(a + h * n) * h;

   cout << "____________________\n";
   cout << "|Result is>>>>>>>-<" << "|" << "\n";
   cout << "|Left:::::" << Integral[0] << "<|" << "\n";
   cout << "|Middle:::" << Integral[1] << "<|" <<"\n";
   cout << "|Right::::" << Integral[2] << "<|" <<"\n";
   cout << "|Trapeze::" << Integral[3] << "<|" <<"\n";
   cout << "--------------------\n";
   return 0;
}
    \end{lstlisting}
    
    \begin{itemize}
    
    	\item Метод левых прямоугольников : 0.626627; $\delta = 0.00003$
    	\item Метод центральных прямоугольников : 0.626622; $\delta = 0.000035$
    	\item Метод правых прямоугольников : 0.626617;  $\delta = 0.00004$
    	\item Метод трапеций : 0.626622;  $\delta = 0.000035$
    	
    \end{itemize}
    
        \subsection*{3 задание}
    
    \addcontentsline{toc}{subsection}{3 задание}
    Для следующих дифференцильных уравнений определить тип, и с помощью программ компьютерной математики найти общее решение:
    
    \begin{enumerate}
    	
    	\item $y \cdot \ln{y}-y'\cdot \sqrt{1-x^2} = 0$\\  
    	
    	\textbf{Решение:} 
    	
    	Уравнение первого порядка с разделяющимися переменными 
    	
    	\[\log \log y=\arcsin x+{\it C.}\]\\
    	
    	
    	\item $xy' = \displaystyle \frac{\sec{xy}}{y} - y$ \\
    	
    	\textbf{Решение:} 
    	
    	Обобщенное однородное уравнение первого порядка    
    	
    	\[-\, \frac{1}{x^{2}} + x\,y \cdot \sin{x\,y} + \cos{x\,y} = {\it C.}  \]	\\	
    	
    	\item $y'=\displaystyle\frac{x+y}{y-x+2}$\\
    	
    	\textbf{Решение:} 
    	
    	Уравнение первого порядка, сводящееся к однородному 
    	
    	\[{{y^2+\left(4-2\,x\right)\,y-x^2}\over{2}}={\it C.}\]\\
    	\item $y'\cdot\sin{x} + y \cdot \cos{x} = \tg{x}$ \\
    	
    	\textbf{Решение:} 
    	
    	Линейное уравнение первого порядка 
    	
    	\[y={{\ln{\sec{x}}+{\it C}}\over{\sin x}}.\]\\
    	
    \end{enumerate}

    \subsection*{4 задание}
    \addcontentsline{toc}{subsection}{4 задание}
    Проверить, является ли представленная неявная функция решением следующей задачи Коши:
    \[y' \cdot (2y - x \cdot \sqrt{e^{2y^{2}}-1}) = y \cdot \tg~xy,~ y(0) = 1; ~ y^2 = \ln~\sec~xy\]
    
    \begin{verbatim}
    
        (%i1)	solution(x, y) := y^2 = log(sec(x*y));
        (%o1)	solution(x,y):=y^2=log(sec(x*y))
        (%i2)	solution(0, 1);
        (%o2)	1=0
        
    \end{verbatim}
    
	Условие не выполняется, следовательно представленная неявная функция не является решением данной задачи Коши.
    
\end{document}